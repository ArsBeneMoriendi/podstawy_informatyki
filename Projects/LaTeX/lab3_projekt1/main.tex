\documentclass{article}
\usepackage{graphicx}
\usepackage[utf8]{inputenc}
\usepackage[polish]{babel}
\usepackage[T1]{fontenc}

\title{\emph{\textbf{Omega} - akademicka klasa regatowa}}
\author{Zuzanna Dybcio}
\date{}

\begin{document}
\maketitle
\tableofcontents
~\\

\begin{figure}[h]
    \centering 
    \includegraphics[width=\textwidth]{woda.jpg}
    \caption{Omegi, należące do AZS AGH, przycumowane do pomostu na czas przerwy} 
\end{figure}

\section*{Wprowadzenie}
Jako iż w tę sobotę byłam na treningu swojej sekcji AZS - \underline{sekcji żeglarskiej}, to postanowiłam napisać artykuł na temat jachtów, na których pływamy, oraz regat organizowanych w ramach \textbf{\emph{Akademickich Mistrzostw Polski}}.

\section{Krótko o Omegach}
\textbf{Omega} jest popularną w Polsce klasą jachtów regatowych do żeglugi śródlądowej. Ma ożaglowanie \emph{bermudzkie} (trójkątne żagle) typu \emph{slup}, tzn. ma jeden maszt, na nim przymocowany likiem przednim \textbf{grot}, a z przodu, na sztagu, jest obecny \textbf{fok}.\\
Pierwsze Omegi powstały w Warszawie podczas okupacji (\textbf{w 1942 r.}). Ich konstruktorem był \textbf{Juliusz Sieradzki}, absolwent Politechniki Warszawskiej, który swoją żeglarską przygodę zaczął na Wiśle w warszawskim AZS. Jacht zaprojektowano jako drewniany z poszyciem słomkowym i tak wytwarzany był przez wiele lat, jednak w XXI wieku drewniane egzemplarze są już raczej rzadko spotykane. Obecnie popularne są Omegi \underline{wykonane z laminatu poliestrowo-szklanego.}

\section{Dane w pigułce}
\begin{enumerate}
    \item Typ: slup
    \begin{itemize}
        \item ożaglowanie bermudzkie
        \item żagle:
        \item powierzchnia całkowita: 18 m²
        \item grot: 11,7 m²
        \item fok: 6,3 m²
        \item spinaker: ok. 20 m² dopuszczony do regat od 2001
        \item oznaczenie na grocie: $\Omega$, czarna, duża grecka litera omega
    \end{itemize}
    \item Wymiary:
    \begin{itemize}
        \item długość całkowita: 6,15 - 6,25 m
        \item szerokość całkowita: 1,75 - 1,85 m
        \item zanurzenie: 0,16 m
        \item zanurzenie z mieczem: 0,96 m
        \item wysokość masztu: 8,30 m
    \end{itemize}
    \item Masa całkowita: 320 kg (minimalna 250 kg)
    \item Załoga: 3 - 6, (w czasie oficjalnych regat - 3)
\end{enumerate}

\section{Akademickie Mistrzostwa Polski i osiągnięcia AGH}
Pierwsze żeglarskie AMPy odbyły się w \textbf{1956 roku}, a \textbf{od 1974 do teraz} odbywają się \textbf{co rok} (z wyjątkiem 1977 i 1978), zawsze w \emph{Wilkasach} na jeziorze \emph{Niegocin}.\\
Jednak \textbf{AGH} w całej historii AMPów zdobyła \underline{tylko jeden medal} (\emph{brązowy medal} w 2019 - załoga \emph{Aleksandra Michalskiego}, który to w latach 2020-2022 zdobywał \emph{złote i srebrne medale} dla Politechniki Gdańskiej). \textbf{Czas to zmienić ;)}

\begin{figure}[h]
    \centering 
    \includegraphics[width=\textwidth, angle=270]{brzeg.jpg}
    \caption{Złożone Omegi po treningu} 
\end{figure}

\begin{figure}[h]
    \centering 
    \includegraphics[width=\textwidth]{zachod.jpg}
    \caption{A tu na dodatek piękny widoczek z Mostu Powstańców Śląskich w czasie powrotu rowerem z treningu} 
\end{figure}

\end{document}
